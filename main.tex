%%%%%%%%%%%%%%%%%%%%%%%%%%%%%%%%%%%%%%%%%%%%%%%%%%%%%%%%%%%%%%%%%%%%%%%%%%%%%%%
% Cookbook
% Cookbook of original, adapted and copied recipes
% Authors:
% Frederick Watkins
%%%%%%%%%%%%%%%%%%%%%%%%%%%%%%%%%%%%%%%%%%%%%%%%%%%%%%%%%%%%%%%%%%%%%%%%%%%%%%%

\documentclass[a4paper, oneside]{book}

\usepackage{amssymb}
\usepackage{tabularx}
\usepackage[UKenglish]{babel}
\usepackage{multicol}
\usepackage{colortbl}
\usepackage{utfsym}
\usepackage{fontawesome}
\usepackage{titlesec}
\usepackage{titletoc}

\titleformat
{\chapter} % command
[display] % shape
{\Huge\bfseries} % format
{\centering Chapter \thechapter .} % label
{0.5ex} % sep
{
    \vspace{1ex}
    \centering
} % before-code
[
] % after-code

\titleformat
{\section} % command
[display] % shape
{\huge\bfseries} % format
{} % label
{0.5ex} % sep
{
} % before-code
[
] % after-code

\titlecontents{section}[2em]
{}%
{}% numbered sections formatting
{\itshape}% unnumbered sections formatting
{{\hspace{0.4em}\titlerule*[6pt]{.}}\contentspage}%


\titleformat
{\subsection} % command
[display] % shape
{\Large\bfseries} % format
{} % label
{0.5ex} % sep
{
} % before-code
[
    \vspace{2ex}
] % after-code

\titlecontents{subsection}[4em]
{}%
{}% numbered sections formatting
{\itshape}% unnumbered sections formatting
{{\hspace{0.4em}\titlerule*[6pt]{.}}\contentspage}%

\newcommand{\gray}{\rowcolor[gray]{.90}} 

\title{\Huge\textbf{cookbook}}
\author{Frederick Watkins}

\begin{document}

\selectlanguage{UKenglish}

\maketitle

\tableofcontents

\chapter{North American}

\section{Meat}

\subsection{BBQ Beef Burger}

\begin{tabularx}{\linewidth}{*{2}{X}}
    \gray \usym{1F551}\space Preperation time: & \textbf{10 minutes}\\
    \gray \faFire\space Cooking time: & \textbf{10 minutes}\\
    \gray \faUser\space Serves: & \textbf{4 people}\\
\end{tabularx}

\begin{multicols}{2}
        Equipment:
        {\begin{itemize}
            \item Burger press
            \item BBQ (Ideally coal)
        \end{itemize}}
        Ingredients:

        For the burger patty:
        {\begin{itemize}
            \item 400g minced beef
            \item 1 egg (Beaten but not aerated)
            \item Rosemary
            \item Thyme
        \end{itemize}}
        For the burger:
        {\begin{itemize}
            \item Brioche/sourdough bun (personal preference)
            \item Emmenthal cheese (sliced)
            \item Lettuce
            \item Tomato
        \end{itemize}}
    Instructions:
    {\begin{enumerate}
        \item
            Cut the lettuce into pieces roughly the size of the buns, and cut
            the tomatoes into roughly 0.5cm thick slices.
        \item 
            Put the minced beef in a bowl with the egg, herbs and mash it
            together with your hands until it easily holds a shape.
        \item
            Split the meat into roughly palm sized evenly sized balls.
        \item
            Put the balls into a burger press and press until they are roughly
            1cm thick, or until the patties reach the edge of the press and are
            even.
        \item
            Gently place the patties onto a hot barbeque, being careful not to
            split them or drop any meat through the grill.
        \item
            Flip the patties one the underside of them browns. Once again, be
            careful not to split them or drop meat through the grill.
        \item 
            Once the patties are brown on both sides, place each in a bun with
            a slice of cheese, lettuce and some slices of tomato on top.
    \end{enumerate}}

\end{multicols}

\subsection{Chimichanga}

\subsection{New England Chop Suey}

\subsection{Borracho Beans}

\subsection{Kansas City Barbeque}

\subsection{Laulau}

\section{Poultry}

\subsection{Buttermilk Chicken Burger}

\subsection{Chicken and Waffles}

\subsection{Nashville Hot Chicken}

\section{Seafood}

\subsection{Louisiana Crawfish Boil}

\subsection{Crawfish Étouffée}

\section{Vegetarian}

\subsection{Mac 'n' Cheese}

\subsection{Tex-Mex Nachos}

\subsection{Eggs Benedict}

\chapter{British}
\section{Meat}
\subsection{Beef Wellington}
\subsection{Beef, Tomato, and Mushroom Stew}

\chapter{Spanish}
\section{Seafood}

\subsection{Shellfish Paella}

\begin{tabularx}{\linewidth}{*{2}{X}}
    \gray \usym{1F551}\space Preperation time: & \textbf{10 minutes}\\
    \gray \faFire\space Cooking time: & \textbf{60 minutes}\\
    \gray \faUser\space Serves: & \textbf{8 people}\\
\end{tabularx}

\begin{multicols}{2}
    Equipment:

    {\begin{itemize}
        \item Paella pan/large frying pan
        \item Saute pan/small frying pan
        \item Medium saucepan
    \end{itemize}}
    Ingredients:
    {\begin{itemize}
        \item 220g small clams
        \item Salt
        \item 220g mussels
        \item 6 tbsp olive oil
        \item 1 green bell pepper
        \item 450g cuttlefish or squid, cut into 5cm by 1cm pieces
        \item 18 large head-on shrimp with tails
        \item 6 head on prawns
        \item 4 medium tomatoes, finely chopped
        \item \(\frac{1}{2}\) tsp sweet pimenton
        \item 2 pinches saffron
        \item \(\frac{1}{4}\) cup peas
        \item \(2\frac{1}{2}\) cup short or medium grain rice
        \item 1 lemon
    \end{itemize}}
    Instructions:
    {\begin{enumerate}
        \item 
            Bring five cups of salted water to a boil and add the clams. Reduce
            to a simmer and cover for 30 mins. Remove from the heat and set
            aside.
        \item
            Put the mussels in a small saute pan with \(\frac{1}{3}\) cup of
            water and bring to a boil. Simmer uncovered until all of the
            mussels have opened or for around 3 to 5 minutes. Set aside covered
            (do not drain).
        \item
            In the paella pan, heat the oil over medium heat. Add the bell
            pepper and cook while stirring until it begins to brown.
        \item
            Add the cuttlefish and cook until browned, stirring constantly.
        \item
            Lower the heat and add the tomatoes and 2 pinches of salt
            and cook, occasionally stirring, until the tomatoes have darkened.
        \item
            Sprinkle in the pimenton and saffron while stirring constantly.
            Stir 4 cups of the clam broth and 2 cups of water into the paella,
            discard any unopened clams and the empty side of each clam shell,
            and then add the clams and peas.
        \item
            Simmer while tasting and adjusting the seasoning as needed.
        \item 
            Increase the heat and bring the mixture to a boil and add the
            shrimp and the prawns, and sprinkle the rice into the pan.
        \item
            Evenly distribute the rice and then do not stir until the rice
            is both cooked and dry. Add more liquid if the rice dries but is
            still undercooked.
        \item
            Drain the mussels and discard the empty side of each shell. Lay the
            mussles on top of the rice in a decorative concentric pattern.
            Serve with a slice of lemon.
    \end{enumerate}}

\end{multicols}

\chapter{Italian}

\section{Meat}

\subsection{Beef and Olive Papadelle}

\section{Poultry}

\subsection{Chicken Parmigiana}

\begin{tabularx}{\linewidth}{*{2}{X}}
    \gray \usym{1F551}\space Preperation time: & \textbf{20 minutes}\\
    \gray \faFire\space Cooking time: & \textbf{20 minutes}\\
    \gray \faUser\space Serves: & \textbf{2 people}\\
\end{tabularx}

\begin{multicols}{2}
    Equipment:
    {\begin{itemize}
        \item Medium saucepan
        \item Large baking tray
        \item Glass oven-safe dish
    \end{itemize}}
    Ingredients:

    For the chicken breast:
    {\begin{itemize}
        \item 2 medium chicken breasts
        \item 2 eggs (beaten but not aerated)
        \item 75g breadcrumbs
        \item 75g grated parmesan
        \item Olive oil
        \item 2 crushed garlic cloves
        \item \(\frac{1}{2}\) mozzarella ball, torn
    \end{itemize}}
    For the sauce:
    {\begin{itemize}
        \item 350ml passata
        \item 1tsp dried oregano
    \end{itemize}}
Instructions:
{\begin{enumerate}
    \item 
        Bring the passata and oregano to a simmer in a saucepan for 10 mins,
        then set aside.
    \item
        Butterfly the chicken breasts by placing the straight edge of the
        breast in your palm and cutting towards it with the blade of the knife
        aligned with the breast, cutting almost to the edge but leaving it
        intact.
    \item 
        Open the chicken breasts and put a sheet of clingfilm on top of them.
        Beat them until thin and tender using a rolling pin.
    \item 
        Set up three plates for breadcumbing the chicken. On the first plate,
        pour the beaten egg, on the second the breadcumbs, and on the third
        half of the parmesan. Coat the chicken breasts in that order and place
        on an oiled baking tray under the grill on high.
    \item 
        Grill the chicken for 5 minutes on each side, and then remove and place
        on the glass dish on top of a bed of the sauce. Sprinkle the remaining
        parmesan and the mozzarella on top of the chicken.
    \item 
        Grill for a further 5 minutes, or until the cheese has completely
        melted.
\end{enumerate}}

\end{multicols}

\section{Seafood}

\subsection{Spaghetti Alle Vongole}

\section{Vegetarian}

\subsection{Mascarpone and Truffle Pizza}

\subsection{Tomato and Mascarpone Risotto}

\chapter{German}

\end{document}